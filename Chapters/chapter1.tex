\chapter{Εισαγωγή}
\label{chap:Intro}

TODO: Θέλει ανανέωση.

Την τελευταία δεκαετία έχει σημειωθεί σημαντική πρόοδος σε 
συστήματα που εκμεταλλεύονται την αρχιτεκτονική peer-to-peer.

Ένα δίκτυο ομότιμων (peer-to-peer) $[$παραπομπή$]$ είναι μια 
αρχιτεκτονική βάσει της οποίας κάθε μέλος (peer) έχει ισότιμες 
δυνατότητες με τα υπόλοιπα. Στα αμιγώς peer-to-peer δίκτυα, δεν υπάρχει 
κάποιος κόμβος που να κατευθύνει και να οργανώνει τους peer. Μάλιστα, 
κάθε peer συμμετέχει ισοδύναμα στην οργάνωση και στη λειτουργία του 
δικτύου προσφέροντας μέρος των πόρων του. 

Ένα από τα σημαντικά προβλήματα που συναντάμε σε τέτοιες αρχιτεκτονικές, 
είναι η αστοχία των peer κατά τη διάρκεια ζωής του δικτύου. Ένας peer 
μπορεί να αποτύχει για δυο λόγους, είτε έχει πρόβλημα ο ίδιος (έχει 
κρασάρει κλπ), είτε υπάρχει πρόβλημα στην σύνδεση του. Οπότε σημαντικό 
χαρακτηριστικό είναι να συνεχίσει ένα peer-to-peer σύστημα να λειτουργεί 
σωστά καθώς και οι υπηρεσίες που παρέχονται να είναι διαθέσιμες 
(available). Υπάρχει η ανάγκη ενός πρωτοκόλλου βάσει του οποίου 
προλαμβάνεται το προηγούμενο πρόβλημα. Ένας μηχανισμός, ο οποίος να 
είναι σε θέση να αναγνωρίσει γρήγορα και με ακρίβεια peer που αστοχούν ή 
αποκλίνουν από την αναμενόμενη συμπεριφορά. Γενικά, ένας peer μπορεί να 
μαθαίνει πληροφορίες για τους υπόλοιπους με δυο τρόπους $[$παραπομπή$]$. 
Μπορεί να ρωτήσει απευθείας τους peer που τον ενδιαφέρουν άμεσα, όπως 
αυτοί που είναι αποθηκευμένοι στο routing table του, περιμένοντας κάποια 
απάντηση από αυτούς ως απόδειξη ότι λειτουργούν σωστά. Επίσης, από τη 
στιγμή που κάποιοι peer απλά δρομολογούν τα μηνύματα, δεδομένου ότι δεν 
έχουν φτάσει ακόμα στον τελικό παραλήπτη, μπορεί να γίνει γνωστό στο 
δίκτυο ποιοι peer είναι ενεργοί. Χρησιμοποιώντας αυτές τις πληροφορίες 
οι peer μπορούν να διορθώσουν τις προβληματικές αναφορές. Μεγάλο ρόλο 
παίζει ο χρόνος που χρειάζεται για να αναγνωριστεί μια αστοχία. Όσο πιο 
γρήγορα εντοπιστεί το πρόβλημα τόσο πιο γρήγορα θα διορθωθεί κι άρα το 
δίκτυο ξοδεύει λιγότερους πόρους για να στείλει μηνύματα σε peer που 
έχουν πέσει. Παρόμοια μοντέλα έχουν περιγραφεί στις δημοσιεύσεις 
$[$παραπομπή$]$ και $[$παραπομπή$]$.

Το P-Grid $[$παραπομπή$]$, είναι ένα πρωτόκολλο το οποίο καθορίζει τον 
τρόπο οργάνωσης και αλληλεπίδρασης των peer. Ανήκει στην κατηγορία των 
δομημένων δικτύων ομότιμων (structured peer-to-peer). Οργανώνεται σε ένα 
κατανεμημένο κατακερματισμένο πίνακα (DHT) και η τοπολογία που 
δημιουργεί είναι όμοια με αυτή ενός binary trie. Είναι ένα πλήρως 
αποκεντρωμένο δίκτυο που βασίζεται σε τυχαιοκρατικούς αλγορίθμους για 
την κατασκευή των πινάκων δρομολόγησης (routing table). Τη δεδομένη 
στιγμή, το πρωτόκολλο P-Grid, υλοποιεί ένα αλγόριθμο αντιγραφής 
δεδομένων (replication). Υπάρχουν, συνεπώς, οι κόμβοι οι οποίοι έχουν 
την αρχική πληροφορία και οργανώνουν το DHT και οι κόμβοι οι οποίοι 
αντιγράφουν την πληροφορία και επιλέγονται τυχαία από τους 
προηγούμενους. Με replication επιτυγχάνεται η σταθερότητα του συστήματος 
και μάλιστα αυξάνεται η πιθανότητα να βρεθεί ένα κλειδί, μιας και 
συμμετέχουν στην αναζήτηση όλοι οι peer. Πρέπει, όμως, οι replicated 
peer να συγχρονίζουν διαρκώς τα δεδομένα τους ώστε να έχουμε data 
consistency. Η διαδικασία αυτή στο P-Grid γίνεται ανά τυχαία χρονικά 
διαστήματα. Στη δημοσίευση $[$παραπομπή$]$, αναφέρονται γενικά 
προβλήματα που μπορεί να υπάρξουν με την τεχνική του replication.

Αντίστοιχα, στο Chord $[$παραπομπή$]$ οι peer ανήκουν σε ένα κύκλο 
κρατώντας ο καθένας πληροφορίες για τους successor τους. Κάθε peer 
αποφασίζει ποια είναι η κατάσταση του γείτονά του πχ κάνοντας ping. Σε 
περίπτωση που καταλάβει κάποια αστοχία τότε βρίσκει το νέο successor του 
(διαδικασία stabilization) και διορθώνει κατάλληλα το routing table του. 
Το ζητούμενο είναι να διατηρηθεί ο κύκλος. Δεν έχουμε data consistency 
σε επίπεδο πρωτοκόλλου και η αποτυχία κόμβου ισοδυναμεί με απώλεια 
δεδομένων. Επίσης, ένα άλλο πρόβλημα που προκύπτει όταν έχουμε 
συνεχόμενες αποτυχίες είναι η δημιουργία πολλαπλών κύκλων.

Σε άλλα πρωτόκολλα όπως το Tapestry $[$παραπομπή$]$ έχουμε redundancy. 
Τα μηνύματα ακολουθούν συγκεκριμένο μονοπάτι και κάθε peer αποθηκεύει 
έναν αριθμό από δείκτες προς άλλους peer. Στην περίπτωση αστοχίας ενός 
peer που άνηκε στο μονοπάτι, επιλέγεται ένας από τους προηγούμενους 
δείκτες και προωθείται το μήνυμα εκεί. Εδώ, έχουμε ένα επιπλέον overhead 
για την αποθήκευση των redundant δεικτών καθώς και επιπλέον maintenance.
