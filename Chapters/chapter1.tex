\chapter{Εισαγωγή}
\label{chap:Intro}

\section{Πρόβλημα και Πλαίσιο Έρευνας}

Τα peer-to-peer συστήματα έχουν αναδειχθεί τα τελευταία χρόνια όχι μόνο 
ως ερευνητικό αντικείμενο αλλά και ως ένα επιτυχημένο μοντέλο ανάπτυξης 
εφαρμογών. Αρχικά έγιναν γνωστά από εφαρμογές διαμοιρασμού αρχείων. Στην 
πορεία όμως αναπτύχθηκαν συστήματα σε ευρύτερο φάσμα εφαρμογών. 
Συγκεντρωτικά υπάρχουν διάφοροι τομείς όπου εκμεταλλεύεται το μοντέλο 
των peer-to-peer συστημάτων όπως αναφέρεται στην 
\citep{Androutsellis-Theotokis2004}. Έχουμε συστήματα στον τομέα της 
επικοινωνίας (communication \& collaboration) όπως chat και υπηρεσίες 
άμεσων μηνυμάτων (instant messaging). Παραδείγματα όπως το σύστημα Seti@home 
από τον χώρο του κατανεμημένου υπολογισμού (distributed computation). 
Έχουν αναπτυχθεί συστήματα που παρέχουν υπηρεσίες υποστήριξης όπως για 
παράδειγμα εφαρμογές που αφορούν την ασφάλεια συστημάτων. Σημαντική 
εργασία έχει γίνει στον χώρο των κατανεμημένων βάσεων δεδομένων όπως 
NoSQL συστημάτων. Τέλος έχουμε συστήματα διανομής περιεχομένου (content 
distribution) που είναι και η πιο συνηθισμένη εφαρμογή των peer-to-peer 
συστημάτων.

Με την εξέλιξη της τεχνολογίας έχουμε την δημιουργία νέων αναγκών και 
απαιτήσεων όσον αφορά τα τεχνολογικά χαρακτηριστικά που προσφέρουν τα 
peer-to-peer συστήματα. Η αρχιτεκτονική λογισμικού που ακολουθεί ένα 
τέτοιο σύστημα είναι αυτή που θέτει τα όρια στον προγραμματιστή αυτών. 
Την τελευταία δεκαετία έχουν γίνει διάφορες δοκιμές και πειραματισμό 
στην προσπάθεια τυποποίησης αυτών. Η άνοδος του μοντέλου της 
υπηρεσιοκεντρικής αρχιτεκτονικής (Service-Oriented Architecture) μας 
δίνει την έμπνευση και το θεωρητικό υπόβαθρο για την σχεδίαση του 
λογισμικού που πραγματεύεται η παρούσα διπλωματική εργασία.

Ένα δεύτερο ζήτημα που τίθεται είναι η διατήρηση του δικτύου και της 
τοπολογίας των peer-to-peer συστημάτων κατά την διάρκεια ζωής τους. Όπως 
θα δούμε και στην ενότητα \ref{sec:Fault}, αναφερόμαστε σε προβλήματα που 
δημιουργούνται από το φαινόμενο churn και την αποτυχία των peer. Ένας 
peer μπορεί να αποτύχει για δυο λόγους, είτε έχει πρόβλημα ο ίδιος (έχει 
κρασάρει κλπ), είτε υπάρχει πρόβλημα στην σύνδεση του. Οπότε σημαντικό 
χαρακτηριστικό είναι να συνεχίσει ένα peer-to-peer σύστημα να λειτουργεί 
σωστά καθώς και οι υπηρεσίες που παρέχονται να είναι διαθέσιμες. Υπάρχει 
η ανάγκη ενός πρωτοκόλλου βάση του οποίου προλαμβάνεται το προηγούμενο 
πρόβλημα. Ένας μηχανισμός, ο οποίος να είναι σε θέση να αναγνωρίσει 
γρήγορα και με ακρίβεια peer που αστοχούν ή αποκλίνουν από την 
αναμενόμενη συμπεριφορά. Γενικά, ένας peer μπορεί να μαθαίνει 
πληροφορίες για τους υπόλοιπους με δυο τρόπους όπως έχει αναλυθεί και 
στη δημοσίευση \citep{Stoica2005}. Μπορεί να ρωτήσει απευθείας τους peer 
που τον ενδιαφέρουν άμεσα, όπως αυτοί που είναι αποθηκευμένοι στο 
routing table του, περιμένοντας κάποια απάντηση από αυτούς ως απόδειξη 
ότι λειτουργούν σωστά. Επίσης, από τη στιγμή που κάποιοι peer απλά 
δρομολογούν τα μηνύματα, δεδομένου ότι δεν έχουν φτάσει ακόμα στον 
τελικό παραλήπτη, μπορεί να γίνει γνωστό στο δίκτυο ποιοι peer είναι 
ενεργοί. Χρησιμοποιώντας αυτές τις πληροφορίες οι peer μπορούν να 
διορθώσουν τις προβληματικές αναφορές. Μεγάλο ρόλο παίζει ο χρόνος που 
χρειάζεται για να αναγνωριστεί μια αστοχία. Όσο πιο γρήγορα εντοπιστεί 
το πρόβλημα τόσο πιο γρήγορα θα διορθωθεί κι άρα το δίκτυο ξοδεύει 
λιγότερους πόρους για να στείλει μηνύματα σε peer που έχουν πέσει. 
Παρόμοια μοντέλα έχουν περιγραφεί στη δημοσίευση \citep{Rodrigues2002}.

\section{Στόχοι}

Η διπλωματική χωρίζεται σε δυο κομμάτια. Το πρώτο κομμάτι αφορά την 
αρχιτεκτονική του ενδιάμεσου λογισμικού. Το λογισμικό που θα αναπτυχθεί 
θα καθιστά δυνατή τη δημιουργία εφαρμογών που στηρίζονται στο peer-to-peer 
μοντέλο. Η σχεδίαση καθοδηγείται από τις εξής απαιτήσεις:

\begin{itemize}
\item Ένα peer-to-peer σύστημα περιέχει πρωτόκολλα βάσει αυτών οι peer 
αλληλεπιδρούν. Ανάγκη μιας κατηγορίας χρηστών είναι η επέκταση της 
βιβλιοθήκης με νέα πρωτόκολλα καθώς και ο μεταξύ τους συνδυασμός ώστε να 
προκύψουν πιο πολύπλοκα. Η διαδικασία αυτή, δηλαδή η δημιουργία και η 
σύνθεση πολύπλοκων πρωτοκόλλων καθώς και η ευκολία με την οποία 
επιτυγχάνεται αυτό, πρέπει να αποτυπωθεί στην αρχιτεκτονική.
\item Παροχή μηχανισμού ανοχής λαθών.
\end{itemize}

Το δεύτερο κομμάτι της εργασίας είναι η δημιουργία αυτού του μηχανισμού 
ανοχής λαθών. Επιδιώκουμε την βελτιστοποίηση της συμπεριφοράς του 
δικτύου κατά την αποτυχία των κόμβων του (fault-tolerance). Στόχος είναι 
η υλοποίηση πρωτοκόλλου, βάσει του οποίου το δίκτυο να μπορεί να αναγνωρίσει 
αστοχίες peer, να τις διορθώσει και να συνεχίσει να λειτουργεί ομαλά.

\section{Οργάνωση Διπλωματικής}

Στο κεφάλαιο \ref{chap:P2P} εξηγείται το μοντέλο που ακολουθούν τα 
peer-to-peer συστήματα. Επίσης, αναλύεται η δομή και ο τρόπος λειτουργίας 
του P-Grid 
\citep{Abererb, Aberer, Abererc, Abererd, Aberer2004, Aberer2003, Aberere, Aberer2002} 
συστήματος.

Στο κεφάλαιο \ref{chap:Literature} γίνεται μια επισκόπηση της βιβλιογραφίας 
που ασχολείται με την αρχιτεκτονική των peer-to-peer συστημάτων. 
Αναλύεται η αρχιτεκτονική των υπηρεσιοκεντρικών συστημάτων και τον ρόλο 
που παίζει η υπηρεσία ως έννοια σε αυτά. Το συγκεκριμένο μοντέλο 
χρησιμοποιήθηκε ως βάση για την ανάπτυξη της δικής μας αρχιτεκτονικής.

Στο κεφάλαιο \ref{chap:Arch} περιγράφεται η γενική δομή της προτεινόμενης 
αρχιτεκτονικής. Παρουσιάζονται τα διάφορα αρχιτεκτονικά επίπεδα και η 
λειτουργικότητα που ενθυλακώνει το καθένα. Αναλύεται ο τρόπος που λύθηκαν 
κάποια βασικά θέματα και καταλήγει με την γενική παρουσίαση των 
επιπέδων.

Στο κεφάλαιο \ref{chap:Protocols} αναλύονται τα πρωτόκολλα που υλοποιούνται 
στο σύστημα. Το πρώτο είναι τα βήματα που ακολουθούν δυο peer όταν 
επικοινωνούν που είναι η εκτέλεση του αλγορίθμου exchange όπως περιγράφεται 
στην δημοσίευση \citep{Abererb}. Τέλος, αναλύεται το πρωτόκολλο ανοχής λαθών, 
ο τρόπος με τον οποίο οι peer διορθώνουν αποτυχίες στο δίκτυο.

Στο κεφάλαιο \ref{chap:Impl} αναλύεται ο τρόπος με τον οποίο υλοποιήθηκε 
η προτεινόμενη αρχιτεκτονική. Εξηγούνται οι δυνατότητες που πρόσφερε η 
χρήση της CORBA. Επίσης, αναλύονται διεξοδικά τα συστατικά κομμάτια κάθε 
επιπέδου.

Στο κεφάλαιο \ref{chap:Demo} παρουσιάζεται ο τρόπος με τον οποίο αναπτύσσει 
κάποιος εφαρμογές με τη χρήση της βιβλιοθήκης που υλοποιήθηκε. Συγκεκριμένα 
εισάγουμε λειτουργικότητα διαμοιρασμού αρχείων στο σύστημα. Αναλύοντας το 
πρόβλημα εξηγείται ο τρόπος σκέψεις και υλοποίησης της λύσης.

Τέλος, στο κεφάλαιο \ref{chap:Closing} παρουσιάζονται τα συμπεράσματα αυτής 
της εργασίας και δίνονται μερικές ιδέες σαν αντικείμενο μελλοντικής εργασίας.
