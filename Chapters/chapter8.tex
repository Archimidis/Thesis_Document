\chapter{Επίλογος}
\label{chap:Closing}

\section{Συμπεράσματα}

Κύριος στόχος της διπλωματικής εργασίας ήταν η σχεδίαση αρχιτεκτονικής 
ενδιάμεσου λογισμικού και η υλοποίηση της. Απαίτηση ήταν η δημιουργία 
πολύπλοκων πρωτοκόλλων. Αυτό το πετύχαμε μέσω μιας υπηρεσιοκεντρικής 
αρχιτεκτονικής όπου κάθε λειτουργία που προσφέρει το σύστημα αντιμετωπίζεται 
ως υπηρεσία. Η ευκολία σύνθεσης των υπηρεσιών με ελάχιστο κώδικα οδηγεί 
στην δημιουργία πολύπλοκων πρωτοκόλλων και διαδικασιών. Σε σύγκριση με 
την υλοποίηση του P-Grid αλλά και αντίστοιχες υλοποιήσεις άλλων συστημάτων 
που ακολουθούν παρόμοιο αρχιτεκτονικό μοντέλο, κάτι τέτοιο απαιτεί αρκετές 
αλλαγές και προσθήκες από πλευράς κώδικα. 
Επιπλέον η υπηρεσία κρίνουμε πως είναι η σωστή αφαιρετική έννοια που χωρίζει 
το σύστημα σε διακριτές μονάδες. Η κατανόηση του συστήματος είναι πολύ πιο 
εύκολη με αυτόν τον τρόπο.

Ο δεύτερο στόχος της διπλωματικής που ήταν ταυτόχρονα και απαίτηση κατά τη 
φάση σχεδίασης της αρχιτεκτονική ήταν η παροχή ενός μηχανισμού συντήρησης και 
ανοχής λαθών. Το πρωτόκολλο που αναπτύχθηκε επιλύει αυτό το πρόβλημα χωρίς 
την τεχνική της αντιγραφής (replication) διατηρώντας την τοπολογία του 
δικτύου. Η ανθεκτικότητα του δικτύου μπορεί να αυξηθεί με την συνεργασία 
του πρωτοκόλλου μαζί με replication χωρίς αλλαγές. Μπορεί εύκολα να 
δημιουργηθεί μια υπηρεσία replication και να συνθέσουμε μια διαδικασία 
μαζί με την υπηρεσία διόρθωσης όπως έχει περιγραφεί. 

\section{Μελλοντική Εργασία}

Η υλοποίηση του συστήματος με βάση την αρχιτεκτονική που αναλύθηκε 
έχει αρκετές ελλείψεις όσον αφορά την λειτουργικότητα που προσφέρει. 
Η προσθήκη λειτουργικότητας ενός κατανεμημένου συστήματος αποθήκευσης 
είναι σημαντική. Πέρα από την αποθήκευση δεδομένων προσφέρεται και η 
δυνατότητα επερωτήσεων πάνω σε αυτά. Φυσικά έχουμε διάφορες προκλήσεις 
όπως η αναζήτηση δεδομένων σε ένα εύρος κλειδιών όπως έχει περιγραφεί 
στη δημοσίευση \citep{Grasp}.Επίσης, η υλοποίηση πρωτοκόλλου 
εξισορρόπησης φορτίου (load-balancing) είναι απαραίτητο για την παραπάνω 
λειτουργικότητα.

Η σύνθεση των υπηρεσιών σε διαδικασίες μπορεί αναπαρασταθεί με 
ροές εργασίας (workflow) κατά τα πρότυπα των Web Service. 
έχουμε μια καλύτερη περιγραφή της διαδικασίας Οπότε η 
δημιουργία μιας μηχανής εκτέλεσης ροών εργασίας διευκολύνει αρκετά.

Όπως αναλύσαμε στην υλοποίηση της αρχιτεκτονικής, πολλά πράγματα 
επαναλαμβάνονται. Η αυτοματοποίηση δημιουργίας μιας υπηρεσίας, μέσω ενός 
εργαλείου που θα παράγει βασικές κλάσεις και διεπαφές, είναι χρήσιμη.
