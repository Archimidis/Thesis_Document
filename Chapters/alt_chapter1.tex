\chapter{Εισαγωγή}
\label{chap:Intro}

\section{Πρόβλημα και πλαίσιο έρευνας}

\ \ \ \ TODO

\section{Στόχοι}

Η διπλωματική χωρίζεται σε δυο κομμάτια. Το πρώτο κομμάτι αφορά την 
αρχιτεκτονική του ενδιάμεσου λογισμικού. Το λογισμικό που θα αναπτυχθεί 
θα καθιστά δυνατή τη δημιουργία εφαρμογών που στηρίζονται στο peer-to-peer 
μοντέλο. Η σχεδίαση καθοδηγείται από τις εξής απαιτήσεις:

\begin{itemize}
\item Ένα peer-to-peer σύστημα περιέχει πρωτόκολλα βάσει αυτών οι peer 
αλληλεπιδρούν. Ανάγκη μιας κατηγορίας χρηστών είναι η επέκταση της 
βιβλιοθήκης με νέα πρωτόκολλα καθώς και ο μεταξύ τους συνδυασμός ώστε να 
προκύψουν πιο πολύπλοκα. Η διαδικασία αυτή, δηλαδή η δημιουργία και η 
σύνθεση πολύπλοκων πρωτοκόλλων καθώς και η ευκολία με την οποί 
επιτυγχάνεται αυτό, πρέπει να αποτυπωθεί στην αρχιτεκτονική.
\item Παροχή μηχανισμού ανοχής λαθών.
\end{itemize}

Το δεύτερο κομμάτι της εργασίας είναι η δημιουργία αυτού του μηχανισμού 
ανοχής λαθών. Επιδιώκουμε την βελτιστοποίηση της συμπεριφοράς του 
δικτύου κατά την αποτυχία των κόμβων του (fault-tolerance). Στόχος είναι 
η υλοποίηση πρωτοκόλλου, βάσει του οποίου το δίκτυο να μπορεί να αναγνωρίσει 
αστοχίες peer, να τις διορθώσει και να συνεχίσει να λειτουργεί ομαλά.

\section{Οργάνωση Διπλωματικής}

Στο κεφάλαιο \ref{chap:P2P} εξηγείται το μοντέλο που ακολουθούν τα 
peer-to-peer συστήματα. Επίσης, αναλύεται η δομή και ο τρόπος λειτουργίας 
του P-Grid 
\citep{Abererb, Aberer, Abererc, Abererd, Aberer2004, Aberer2003, Aberere, Aberer2002} 
συστήματος.

Στο κεφάλαιο \ref{chap:Literature} γίνεται μια επισκόπηση της βιβλιογραφίας 
που ασχολείται με την αρχιτεκτονική των peer-to-peer συστημάτων. 
Αναλύεται η αρχιτεκτονική των υπηρεσιοκεντρικών συστημάτων και τον ρόλο 
που παίζει η υπηρεσία ως έννοια σε αυτά. Το συγκεκριμένο μοντέλο 
χρησιμοποιήθηκε ως βάση για την ανάπτυξη της δικής μας αρχιτεκτονικής.

Στο κεφάλαιο \ref{chap:Arch} περιγράφεται η γενική δομή της προτεινόμενης 
αρχιτεκτονικής. Παρουσιάζονται τα διάφορα αρχιτεκτονικά επίπεδα και η 
λειτουργικότητα που ενθυλακώνει το καθένα. Αναλύεται ο τρόπος που λύθηκαν 
κάποια βασικά θέματα και καταλήγει με την γενική παρουσίαση των 
επιπέδων.

Στο κεφάλαιο \ref{chap:Protocols} αναλύονται τα πρωτόκολλα που υλοποιούνται 
στο σύστημα. Το πρώτο είναι τα βήματα που ακολουθούν δυο peer όταν 
επικοινωνούν που είναι η εκτέλεση του αλγορίθμου exchange όπως περιγράφεται 
στην δημοσίευση \citep{Abererb}. Τέλος, αναλύεται το πρωτόκολλο ανοχής λαθών, 
ο τρόπος με τον οποίο οι peer διορθώνουν αποτυχίες στο δίκτυο.

Στο κεφάλαιο \ref{chap:Impl} αναλύεται ο τρόπος με τον οποίο υλοποιήθηκε 
η προτεινόμενη αρχιτεκτονική. Εξηγούνται οι δυνατότητες που πρόσφερε η 
χρήση της CORBA. Επίσης, αναλύονται διεξοδικά τα συστατικά κομμάτια κάθε 
επιπέδου.

Στο κεφάλαιο \ref{chap:Demo} παρουσιάζεται ο τρόπος με τον οποίο αναπτύσσει 
κάποιος εφαρμογές με τη χρήση της βιβλιοθήκης που υλοποιήθηκε. Συγκεκριμένα 
εισάγουμε λειτουργικότητα διαμοιρασμού αρχείων στο σύστημα. Αναλύοντας το 
πρόβλημα εξηγείται ο τρόπος σκέψεις και υλοποίησης της λύσης.

Τέλος, στο κεφάλαιο \ref{chap:Closing} παρουσιάζονται τα συμπεράσματα αυτής 
της εργασίας και δίνονται μερικές ιδέες σαν αντικείμενο μελλοντικής εργασίας.

